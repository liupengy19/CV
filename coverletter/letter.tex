\documentclass[11pt, letterpaper]{article}
\usepackage[utf8]{inputenc}
\usepackage[T1]{fontenc}
\usepackage[margin=1in]{geometry}
\usepackage{parskip}
\usepackage{hyperref}
\usepackage{enumitem}
\usepackage{mathptmx}

\pagestyle{empty}

\begin{document}

\begin{center}
  \textbf{\Large Pengyu Liu} \\[0.5em]
  Ph.D. Student $\cdot$ Computer Science Department $\cdot$ Carnegie
  Mellon University \\
  Email: \href{mailto:pengyul@andrew.cmu.edu}{pengyul@andrew.cmu.edu}
  $\cdot$ Phone: +1 412-251-2404 $\cdot$ Website:
  \href{pengyuliu.me}{pengyuliu.me}
\end{center}

\vspace{1em}
\hrule
\vspace{1em}

\today

\vspace{1em}

Dear QuEra Hiring Committee,

I am a third-year Ph.D. student in Computer Science at Carnegie
Mellon University, advised by Professor Umut A. Acar. I am writing to
apply for the \textbf{Internship - Quantum Error Correction Research}
position at QuEra Computing, where I previously interned and whose
work in neutral atom quantum computing aligns closely with my research.

My research focuses on quantum error correction and fault-tolerant
quantum computing. I have published in top venues (ISCA, POPL, PRL,
SPAA) on topics directly relevant to QuEra's research:

\begin{itemize}[leftmargin=*, parsep=0.5em]
  \item \textbf{Quantum Error Correction:} I am currently working with
    Dr. Chen Zhao on designing syndrome extraction circuits and
    decoders for atom loss errors. We doubled the effective distance
    and increased the threshold by up to 40\% in loss-dominated
    regimes with no extra hardware overhead.

  \item \textbf{Compilation for Neutral Atoms:} My work on neutral
    atom compilation includes ``Atomique: A Quantum Compiler for
    Reconfigurable Neutral Atom Arrays'' (ISCA'24), ``Q-Pilot: Field
    Programmable Qubit Array Compilation with Flying Ancillas''
    (DAC'24), and ``Coniq:
    Enabling Concatenated Quantum Error Correction on Neutral Atom
    Arrays'' (QCE'25).

  \item \textbf{Quantum Algorithms:} I published ``Fundamental
    Limitation on the Detectability of Entanglement'' (PRL), providing
    an in-depth analysis of entanglement detection algorithms. I also
    won first place in the ACM/IEEE Quantum Computing for Drug
    Discovery Challenge.
\end{itemize}

Beyond my technical contributions, I have demonstrated leadership by
organizing the Quantum Computer System Lecture Series and Yao Seminar.
I also have extensive collaborative experience, including remote work
with MIT and the University of Maryland.

If I were to rejoin QuEra, I would like to push atom loss decoding
further---either by converting it to general qLDPC decoding with
existing decoders (Relay-BP, MWPF), or by developing new decoders
inspired by correlated-matching or belief-matching. More broadly, I am
eager to tackle any practical QEC challenges on neutral atom platforms.

I am confident that my expertise in neutral atom systems, quantum
error correction, and quantum algorithms would enable me to contribute
meaningfully to this internship.

Thank you for considering my application. I am looking forward to
contributing to QuEra's groundbreaking work in quantum computing!

\vspace{1em}

Sincerely,

\vspace{1em}

Pengyu Liu \\
Ph.D. Student \\
Computer Science Department \\
Carnegie Mellon University

\end{document}

